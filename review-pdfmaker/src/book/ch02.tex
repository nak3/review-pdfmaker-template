\chapter{文字列処理ツール}
\label{chap:ch02}

\section{less と cat}

\subsection{cat -- 1つ以上のファイルを標準出力にダンプします。}

\begin{reviewemlist}
\begin{alltt}
[root@localhost \textasciitilde{}]\# cat /etc/resolv.conf
\# Generated by NetworkManager
nameserver 192.168.1.1
\end{alltt}
\end{reviewemlist}

\subsection{less -- ファイルや標準入力をページ単位で表示します。}

ページを表示させたあとのコマンド

\begin{itemize}
\item /text ... 「text」を検索
\item n または N ...次 または 前のマッチにジャンプ
\item v ... テクストエディタでファイルを開く
\end{itemize}

\begin{reviewcolumn}
\reviewcolumnhead{}{標準入力と標準出力について}
\end{reviewcolumn}

\section{head と tail}

\subsection{head --- ファイル先頭の10行を表示}

\begin{itemize}
\item オプション   -n : 表示する行数を変更します。
\end{itemize}

\subsection{tail --- ファイルの最後の10行を表示します}

\begin{itemize}
\item オプション   -n : 表示する行数を変更します。
\item オプション   -f ファイルに追加される内容を「追跡」する
\end{itemize}

\section{grep}

\chapter{Linux コマンドサンプル}
\label{chap:ch01}

\section{本文について}
\reviewlistcaption{リスト1.2: Linuxコマンドサンプル}
\begin{reviewlist}
\begin{alltt}
 \textdollar{}date -F
\end{alltt}
\end{reviewlist}


2行以上以上空いていても1行空いているのと同様に処理します。

\subsection{見出し}

「=」「==」「===」の後に一文字空白をあけると見出しになります。

\begin{reviewcolumn}
\reviewcolumnhead{}{コラムについて}

見出しの先頭に「[column]」と書くと、そこはコラムになります。

\end{reviewcolumn}

\section{箇条書き}

番号のない箇条書きは「*」を使います。前後に空白を入れて下さい。

\begin{itemize}
\item 1つ目
\item 2つ目
\item 3つ目
\end{itemize}

番号つきの箇条書きには、「1.」「2.」などと書きます。
数字の値は実際には無視され、連番が振られます。

\begin{enumerate}
\item 1つ目
\item 2つ目
\item 3つ目
\end{enumerate}

\section{ソースコードなどのリスト}

リストには「//list」ブロックや「//emlist」ブロックを使います。

\reviewlistcaption{リスト1.2: リストのサンプル}
\begin{reviewlist}
\begin{alltt}
int main(int argc, char **argv) \{
  puts("OK");
  return 0;
\}
\end{alltt}
\end{reviewlist}


文中にリストを書くには「//emlist」になります。

\begin{reviewemlist}
\begin{alltt}
def main
  puts "ok"
end
\end{alltt}
\end{reviewemlist}

\section{画像}

画像は「//image」ブロックを使います。

\begin{reviewdummyimage}
\begin{alltt}
--[[path =  (not exist)]]--
\end{alltt}
\label{image:ch01:imgsample}
画像サンプル
\end{reviewdummyimage}

より詳しくは、
\url{https://github.com/kmuto/review/blob/master/doc/format.rdoc}
を御覧ください。
